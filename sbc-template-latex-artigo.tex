\documentclass[12pt]{article}

\usepackage{sbc-template}
\usepackage{graphicx}
\usepackage{url}
\usepackage[brazil]{babel} 
\usepackage[utf8]{inputenc}
\usepackage{booktabs}
\usepackage[pdftex]{hyperref}
\usepackage{indentfirst}

     
\sloppy

\title{Título do Trabalho: Subtítulo}

\author{Nome do Autor Um\inst{1}, Nome do Autor Dois\inst{1}, \\Nome do Autor Três\inst{1}, Nome do Autor Quatro\inst{1}, \\Nome do Autor N. \inst{1}, Verônica dos Santos Nascimento\inst{2}, Gilton José Ferreira da Silva\inst{2} } 

\address{Departamento de Computação (DCOMP)\\ Universidade Federal de Sergipe
  (UFS)\\
  Av. Marechal Rondon, s/n -- Jardim Rosa Elze -- CEP 49100-000 \\São Cristóvão -- SE -- Brazil \nextinstitute
  Programa de Pós-Graduação em Ciência da Computação - (PROCC)\\Universidade Federal de Sergipe
  (UFS)\\
  Av. Marechal Rondon, s/n -- Jardim Rosa Elze -- CEP 49100-000 \\São Cristóvão -- SE -- Brazil
  \email{autor01@dcomp.ufs.br, autor02@dcomp.ufs.br}
  \email{autor03@dcomp.ufs.br, autor04@dcomp.ufs.br}
  \email{veronica.nascimento@dcomp.ufs.br, gilton@dcomp.ufs.br}
}

\begin{document} 

\maketitle

\begin{abstract}
\textbf {Context:} Appear the work context. \textbf {Objective:} This work aims to ... \textbf {Method:} Describe the methodological procedures used in the research. \textbf {Results:} Discuss the results of this work. \textbf {Conclusions:} Show the conclusions that were found.
\end{abstract}
     
\begin{resumo} 
  \textbf{Contexto:} Aparentar o contexto do trabalho. \textbf{Objetivo:} Este trabalho tem como objetivo ... \textbf{Método:} Descrever os procedimentos metodológicos utilizados na pesquisa. \textbf{Resultados:} Discutir sobre os resultados advindos deste trabalho. \textbf{Conclusões:} Mostrar as conclusões que foram encontradas.
\end{resumo}


\section{Introdução} \label{sec:introducao}
(Contexto)

Esta seção apresenta o contexto no qual o projeto está inserido, destacando o problema prático ou científico que motivou sua concepção. São abordadas as lacunas existentes, a relevância do tema, o público impactado e a justificativa para a adoção de uma abordagem estruturada de gerenciamento de projetos, especificamente o Project Model Canvas, como instrumento de planejamento e comunicação do projeto.

*No mínimo 1 parágrafo


\subsection{Justificativa}
\label{sec:justificativa}
(Por quê?)

A justificativa descreve as necessidades, demandas não atendidas ou oportunidades que fundamentam a proposição do projeto. Esta seção explicita o problema central que se pretende resolver, sustentado por evidências empíricas, dados institucionais, literatura científica ou normativas vigentes. Busca-se demonstrar a relevância científica, social, organizacional ou econômica do projeto, legitimando sua execução.


\section{Objetivos} 
Objetivos... 1 parágrafo;

\section{Benefícios} 

Benefícios... 1 parágrafo;

\section{Descrição das próximas seções} 
Descrição das próximas seções do trabalho ... 1 parágrafo.


\section{O que será feito? } \label{sec:oque}

Breve descrição das próximas subseções...\\

\subsection{Produto}

Descrição do Produto...

\subsection{Trabalhos Relacionados}

Apresentar uma descrição dos trabalhos (artigos semelhantes que foram buscados nas bases de pesquisa).

\subsection{Arquitetura Visão-Modelo 4+1}

Descrição da Arquitetura conforme a Visão-Modelo 4+1.

A arquitetura visão-modelo 4+1 foi desenvolvida por Philippe Cruchten com o objetivo de descrever o funcionamento de sistemas de software e é baseado no uso de múltiplas visões concorrentes.

As visões são usadas para mostrar o sistema sob várias perspectivas, como usuário final, desenvolvedores e gerentes de projetos.

As quatro visões de modelo são: visão lógica (1), visão de desenvolvimento (2), visão de processo (3) e visão física (4). A visão de caso de uso é usada para ilustrar a arquitetura e representa a visão +1.

Visão lógica : Se concentra na funcionalidade que o sistema disponibiliza para o usuároi final. Os diagramas UML usados para representar a visão lógica incluem: Diagrama de classes, Diagrama de comunicação e Diagrama de sequencia. 
Visão de desenvolvimento : Ilustra o sistema do ponto de vista do programador e se preocupa com o gerenciamento de projeto. Esta visão também é conhecida como visão de implementação. Usa o Diagrama de componentes ou Diagrama de pacotes. 
Visão de processo : Permite visualizar as partes dinâmicas do sistema, explicar os processos e como eles se comunicam, focando no comportamento do sistema. A visão de processo se encarrega da concorrência, distribuição, integração, performance e escalabilidade. O Diagrama de atividades é usado nesta visão.
Visão física : Mostra o sistema do ponto de vista do egenheiro. Se preocupa com a topologia dos componentes de software (no contexto físico) assim como a comunicação entre esses componentes. Esta visão também é conhecida como visão de implantação. Os diagramas UML usados para descrever esta visão incluem o Diagrama de implantação.
Visão de caso de uso : Descreve a arquitetura do sistema através do uso de Diagramas de casos de uso. Cada diagrama descreve sequências de interações entre os objetos e processos. São usados para identificar elementos de arquiteturra e ilustrar e validar o design de arquitetura.

Mais informações no link: 
\url{http://www.basef.com.br/old/uml/204-arquitetura-visao-modelo-41}


\subsection{Requisitos}

Descrição dos Requisitos...


\subsection{Telas do Sistema}

Apresentar As telas referente aos fluxos dos processos de negócio;\\

Obs.: Inclua Figuras, Tabelas, Quadros, Códigos, etc...


A Figura \ref{fig:fluxo_sistema} apresenta um fluxo descrevendo o processo principal do sistema. 

\begin{figure}[!ht]
\centering
\includegraphics[width=12cm]
{images/telas.jpg}
\caption{Gráfico de prisma com a extração de dados}
\label{fig:fluxo_sistema}
\end{figure}




\section{Quem fará o projeto}\label{sec:quem}

Breve descrição das próximas subseções...

\subsection{Stakeholders e Fatores Externos}
parceiros externos necessários para alcançar as metas.


\subsubsection{Mapa de Empatia do Tomador de Decisão}

Inserir o mapa de empatia do tomador de decisão e explicar suas dores e necessidades.


\subsection{Equipe}
Apresentar a Equipe e a funão de cada um

\begin{itemize}
  \item Nome do Autor Um: Pesquisa, correções e escrita do manuscrito;
  \item Nome do Autor Dois: Pesquisa, correções e escrita do manuscrito;
  \item Nome do Autor Três: Pesquisa, correções e escrita do manuscrito;
  \item Nome do Autor Quatro: Pesquisa, correções e escrita do manuscrito;
  \item Verônica dos Santos Nascimento: Pesquisa, correções e escrita do manuscrito;
  \item Gilton José Ferreira da Silva: Coordenação do trabalho, correções e direcionamentos da pesquisa.
\end{itemize}

\section{Como o projeto será feito?} \label{sec:Como}
Breve descrição das próximas subseções...\\


\subsection{Premissas}

\subsection{Grupo de entregas}

\subsection{Restrições}


\section{Quando e quanto?} \label{sec:quandoquanto}
Breve descrição das próximas subseções...\\




\subsection{Riscos}

\subsection{Linha do tempo}

\subsection{Custos}


\section{Limitações do Trabalho} \label{sec:ameacas}

Inserir informações a cerca do que não foi apresentado no trabalho.



\section{Considerações Finais} \label{sec:consideracoes}
Descrição das atividades realizadas para atingir o objetivo do trabalho... 2 paragrafo\\
Descrição da metodologia utilizada para atingir o objetivo do trabalho... 1 paragrafo\\
Descrição dos pontos fracos ou que possam inviabilizar a credibilidade do trabalho... 1 paragrafo\\
Descrição das atividades futuras para se concluir o trabalho... 1 paragrafo\\

\section*{Agradecimentos} \label{sec:agradecimentos}

Esta seção tem como objetivo agradecer a todas as pessoas e instituições que ajudaram na pesquisa, mas que não se qualificam para autoria.

Alguns periódicos e eventos exigem que sejam informados os dados refentes ao às organizações que financiaram a pesquisa.







\bibliographystyle{sbc}
\bibliography{referencias}

\end{document}
